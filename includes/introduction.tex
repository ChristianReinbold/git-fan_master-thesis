\chapter{Introduction}

Algebraic group quotients are commonly used when constructing moduli spaces. A na\"{\i}ve moduli problem is a classification problem, that is a collection of objects together with an equivalence relation. Its moduli space sets up a one\=/to\=/correspondence of its points and the equivalence classes. The projective line $\projective^1$ may be seen as a moduli space of the problem of classifying all nonzero points in the complex plane $\complex^2$ such that two points are equivalent iff they are contained in the same line through the origin. The elements of a class are generated by scaling points by a nonzero scalar. In other words, each class is an orbit of an action of the torus $\complex^*$ on the affine space $\complex^2$ without the origin. The moduli space $\projective^1$ is recovered by the orbit space. However, it is not equipped with a geometric structure by default. Thus, we obtain $\projective^1$ as a set but loose all geometric data attached to it as a projective variety. 

\ac{GIT} provides a method for creating algebraic group quotients with a geometric structure such that the quotient map becomes a morphism. In many instances, the classical orbit space cannot be equipped with such a structure. Consider the complex plane $\complex^2$ with the torus action as above. Since the closure of each orbit -- that is a line -- contains the origin, every regular function being constant on orbits has to be constant on all $\complex^2$. On that account, the ring of regular functions of the orbit space would be the ground field $\complex$ and the orbit space would be a single point. Clearly, this is a contradiction. \ac{GIT} circumvents this problem by removing problematic orbits beforehand. The remaining ones form an open subset of \emph{semistable points} with a well behaving quotient that comes with a geometric structure. The quotient is \emph{almost geometric}, meaning that there exists a dense, open subset such that the quotient becomes the orbit space for this subset. In our case, we have to remove the origin from $\complex^2$ in order to obtain the orbit space $\projective^1$ together with its geometric description.

Mumford's construction of subsets of semistable points depends on a lifting of the group action to ample line bundles where different liftings may yield distinct subsets. There exists a quasifan, called \emph{GIT fan}, whose set of cones parametrises all subsets that may be obtained in this fashion. Given an affine variety together with a toric group action $H\acts X$, we focus on the computation of the GIT fan on scalable hardware systems. Basing on the work of \citeauthor{gitfan_symmetry} \cite{gitfan_symmetry}, we develop a cluster enabled application computing all chambers of the GIT fan of $H\acts X$. In doing so, we implement a fan traversal algorithm that is reusable in the context of tropical varieties \cite{tropical_varities} and Gröbner fans \cite[chapter 3]{sturmfels}.

The thesis is structured as follows: In the preliminaries, we cover the mathematical background encompassed by GIT fans. We introduce algebraic group actions and good categorical quotients, outline Mumford's construction of GIT quotients arising from subsets of semistable points and prove that the GIT cones associated to them indeed form a quasifan with convex support. The next chapter elucidates the algorithm we are going to implement. We describe how to express GIT cones in terms of orbit cones and compute the latter ones. Then, we translate the task of computing the GIT fan into a graph traversal problem and show how symmetries in the geometry of $X$ may be exploited in order to reduce the graph's size. Finally, we point out that  \citeauthor{gitfan_symmetry} incorporated an erroneous optimisation which we left out in our implementation. Fortunately, we were able to reproduce the results of their work without relying on the defect.

The fourth chapter covers the technical details of the cluster enabled implementation. Here, we introduce the utilised parallelisation framework \gpispace{} and follow up with the integration of \singular{} code into \gpispace{} applications. Furthermore, we present a high level description of our implementation in terms of Petri nets and document the various features of our application such as incorporating precomputed results. We finish off the chapter with a performance analysis for the computation of the Mori chamber decomposition of the cone of movable divisor classes of $\overline{M}_{0,6}$. The mathematical background of this example is outlined in the final chapter.
