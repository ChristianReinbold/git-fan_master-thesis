\chapter{Concept of the algorithm}
\label{chap:algorithm}

% Proposition: GIT cone is the intersection of all orbit cones containing it.

% Reference to algorithm describing the whole application
\label{algorithm:main}

\section{Computing orbit cones}
Monomial containment test
Moving Cone
\section{Traversing the GIT fan}

%The traversal yields all nodes in this graph and thus a full description of the GIT fan. 

\todo{Theorie aus Kapitel 4 übernehmen}

\begin{algorithm}
	\caption{Computing all neighbours of an GIT cone orbit}
	\label{algo:git_cone_orbitneighbours}
	
	\KwIn{a GIT cone orbit $\O\in\faktor{\Lambda(\mathfrak{a},Q)(k)}{\mathcal{S}}$}
	\KwOut{All GIT cone orbits $\O_\mathcal{N}\in\faktor{\Lambda(\mathfrak{a},Q)(k)}{\mathcal{S}}$ such that $\{\O, \O_\mathcal{N}\}\in E_\mathcal{S}$}
	\BlankLine
	$\mathcal{N} \leftarrow \emptyset$\;
	Choose $\lambda$ such that $\O = \mathcal{S}\lambda$\;
	\For{$\mu$ such that $\{\lambda,\mu\}\in E_{\langle e \rangle}$}{
		$\mathcal{N}\leftarrow \mathcal{N} \cup \{\mathcal{S}\mu\}$\;
	}
	\Return $\mathcal{N}$\;
\end{algorithm}	


\section{Exploiting symmetry}