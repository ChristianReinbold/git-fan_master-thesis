\begin{center}
	\Large \textbf{Abstract}
\end{center}

The GIT fan of an algebraic group action on an algebraic variety describes all GIT quotients arising from Mumford's construction in Geometric Invariant Theory. In this thesis we enhance an implementation that computes the GIT fan for toric actions on affine varieties such that it may be executed on scalable hardware systems. For this purpose, we combine the parallelisation framework \gpispace{}, developed at the \acl{Fraunhofer ITWM}, and the computer algebra system \singular{}, developed at the \emph{Technische Universität Kaiserslautern}. In doing so, we are able to compute the Mori chamber decomposition of $\Mov(\overline{M}_{0,6})$ in approximately 21 minutes, utilising 40 \emph{Dell PowerEdge M620 Blade Servers} with 16 cores each. The previous implementation executed on a single machine terminated after a whole day.

%\cleardoublepage

\begin{center}
\Large \textbf{Kurzzusammenfassung}
\end{center}

Der GIT\=/Fächer einer algebraischen Gruppwirkung auf einer algebraischen Varietät beschreibt sämtliche GIT\=/Quotienten, die aus Mumfords Konstruktion im Bereich der geometrischen Invariantentheorie hervorgehen. In dieser Arbeit erweitern wir eine bereits bestehende Implementation zur Berechnung des GIT\=/Fächers für torische Gruppenwirkungen auf affinen Varietäten so, dass sie auf beliebig großen Rechnerverbunden parallel ausgeführt werden kann. Hierzu kombinieren wir das am \acl{Fraunhofer ITWM} entwickelte Parallelisierungsframework \gpispace{} mit dem Computeralgebrasystem \singular{} der \emph{Technischen Universität Kaiserslautern}. Mit der neuen Implementation können wir die Zerlegung von $\Mov(\overline{M}_{0,6})$ in seine Mori\=/Kammern in rund 21 Minuten berechnen. Verwendet werden hierbei 40 \emph{Dell PowerEdge M620 Blade Server} mit je 16 Kernen. Die ursprüngliche, auf einer Maschine ausgeführte Implementation terminiert nach einem Tag.