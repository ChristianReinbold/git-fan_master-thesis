\chapter{Conclusion}

%Ausblick: Speed up simplex orbit computation (derzeit noch mit GAP)
% 		   Skalierbare Speicherstruktur mit distributed hash tables
Let $p = x_1x_3+x_2x_4+x_5\in \mathbb{K}[x_1,x_2,x_3,x_4,x_5] =: \mathbb{K}[\mathbf{x}]$. Set 
$$\mathfrak{a} := \langle x_3, x_4\rangle \cdot \langle p\rangle,$$
$$Q = (q_1, q_2, q_3, q_4, q_5) := \begin{pmatrix}
0 & 0 & 1 & 1 & 1 \\
1 & 0 & 0 & 1 & 1 \\
1 & 1 & 1 & 1 & 2 
\end{pmatrix}.$$

Note that $Q$ defines a $\mathbb{Z}^3$ grading on $\faktor{\mathbb{K}[x_1,x_2,x_3,x_4,x_5]}{\mathfrak{a}}$ since $p$ and therefore $\mathfrak{a}$ is homogenous with respect to the grading $\deg(x_i) = q_i$. Thus, we obtain a torus action of $(\mathbb{K}^*)^3$ on $V(\mathfrak{a})$.

The $\mathfrak{a}$-faces with full dimensional image in $Q(\gamma)$, $\gamma = \mathbb{Q}^5_{\geq 0}$, are given by
\begin{align*}
\gamma_1 &:=\langle e_1, e_2, e_5\rangle \\
\gamma_2 &:=\langle e_1, e_2, e_3, e_5\rangle \\
\gamma_3 &:=\langle e_1, e_2, e_4, e_5\rangle \\
\gamma_4 &:=\langle e_1, e_3, e_4, e_5\rangle \\
\gamma_5 &:=\langle e_2, e_3, e_4, e_5\rangle \\
\gamma_6 &:=\langle e_1, e_2, e_3, e_4\rangle \\
\gamma_7 &:=\langle e_1, e_2, e_3, e_4, e_5\rangle
\end{align*}

This is easily seen by sending all variables $x_i$ with $e_i\notin \gamma_j$ in $\mathfrak{a}$ to zero and check that the new ideal does not contain any monomials. (details see below)

The orbit cones $Q(\gamma_j)\in \mathbb{Q}^3_{z_1,z_2,z_3}$ have the following form when intersecting them with the $\{z_3 = 1\}$-plain (vice versa, the orbit cones are recovered by expanding the polytopes in $z_3$-direction to cones in $\mathbb{Q}^3$):

%\psset{xunit=2cm,yunit=2cm}
%\begin{minipage}{.33\textwidth}\centering
%	\begin{pspicture}(-0.5, -0.5)(1.25, 1.25)
%	\psaxes{->}(0,0)(1.25,1.25)
%	\psgrid[griddots=20,subgriddots=10,subgriddiv=2,gridlabels=0pt](2,2)
%	\pspolygon[fillstyle=solid,fillcolor=red,opacity=0.4](0,1)(0,0)(0.5,0.5)
%	\rput(0.6,0.75){$Q(\gamma_1)$}
%	\end{pspicture}
%\end{minipage}
%\begin{minipage}{.33\textwidth}\centering
%	\begin{pspicture}(-0.5, -0.5)(1.25, 1.25)
%	\psaxes{->}(0,0)(1.25,1.25)
%	\psgrid[griddots=20,subgriddots=10,subgriddiv=2,gridlabels=0pt](2,2)
%	\pspolygon[fillstyle=solid,fillcolor=orange,opacity=0.4](0,1)(0,0)(1,0)
%	\rput(0.35,0.25){$Q(\gamma_2)$}
%	\end{pspicture}
%\end{minipage}
%\begin{minipage}{.33\textwidth}\centering
%	\begin{pspicture}(-0.5, -0.5)(1.25, 1.25)
%	\psaxes{->}(0,0)(1.25,1.25)
%	\psgrid[griddots=20,subgriddots=10,subgriddiv=2,gridlabels=0pt](2,2)
%	\pspolygon[fillstyle=solid,fillcolor=orange,opacity=0.4](0,1)(0,0)(1,1)
%	\rput(0.35,0.7){$Q(\gamma_3)$}
%	\end{pspicture}
%\end{minipage}