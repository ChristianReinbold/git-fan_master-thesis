\chapter{Implementation}

In the following we present the implementation of the algorithm discussed in the previous chapter. In contrast to the existing \singular{} implementation \gitfanlib{} \cite{gitfanlib}, which forms the foundation of our work, the approach described here utilises high performance computing methods in order to speed up the execution by running the algorithm on any number of machines simultaneously. Naturally, one has to identify independent components permitting concurrent execution on the one hand and inherent sequential processes on the other hand. As the sequential components of the algorithm mostly agree with \gitfanlib{}, we discuss only reworked parts and skip the remaining ones, referring the interested reader to \cite{gitfanlib}. The chapter is structured as follows: First, we discuss the employed parallelisation framework \gpispace{} developed by the \ac{Fraunhofer ITWM} and follow up with the integration of singular code into \gpispace{} applications. After encoding GIT cone orbits, we present a parallel design of algorithm~\ref{algorithm:main} that is applicable for any kind of fan traversals. Finally, we describe input and output formats, additional features such as speeding up the execution by precomputed results, and executed software tests in order to verify the functional correctness of the program.

\section{\gpispace}

\gpispace{} is a workflow management system developed by the \ac{Fraunhofer ITWM} which supports the execution of arbitrary workflows on ultra scale systems \cite{gpispace}. It consists of three key components: 
\begin{itemize}
	\item the \ac{DRTS}, which is responsible for building and managing arbitrary worker topologies based on the available compute nodes, resource management and job scheduling. It supports the reallocation of jobs in case of hardware failures and further integrates dynamically added hardware
	\item the \ac{WE}. It tracks the state of the workflow, identifies a front of activities for which all dependencies are resolved and laces jobs from input data and active transitions which then are sent to the \ac{DRTS} for scheduling.
	\item a virtual memory layer, which provides the necessary infrastructure for the \ac{PGAS} programming model. It relies on \textsc{GPI} by \ac{Fraunhofer ITWM} \cite{gpi}.
\end{itemize}

The framework has been developed with separation of concerns in mind, whereby the concerns here are given by computation and coordination \cite{coordination_language}. In our case the computation takes place in sequential \singular-routines. The dependencies and data transfers between this routines, which assemble the particular routines into a complex, parallel algorithm, are described by a special domain language in the the coordination layer. In \gpispace{}, this domain language is chosen to be an extension of the well known Petri net model introduced in 1962 by Carl Adam Petri \cite{petri}. Its formal nature permits a vast range of analysis and verification techniques. Concurrency is easily described due to locality of states. Furthermore, Petri nets yield precise execution semantics, supporting interruptions and restarts by differentiating between activation and execution of functional units. For this reason, fault tolerance to hardware failures is achieved by simply restarting failed executions. \cite{petri_net_wms}

\gpispace{} is shipped with an appropriate compiler that generates all files which are necessary for execution from a XML description of a given Petri net. Furthermore, \gpispace{} also provides a basic visualization tool that comes in handy when debugging small to medium sized nets.

\subsection*{Petri nets}

Formally, an (unweighted) \emph{Petri net} is a triple $(P, T, F)$ of \emph{places} $P$, \emph{transitions} $T$ and directed \emph{arcs} $F\subseteq (P\times T)\cup(T\times P)$ relating the former concepts. We demand that $P\cap T = \emptyset$. A function $M\colon P\rightarrow\natural$ is called \emph{marking} and describes a possible state of the Petri net. If we have $M(p)=k$ for a place $p$ and the current state of the Petri net is given by $M$, we say that $p$ holds $k$ \emph{tokens}. Visually, circles depict places, rectangles are transitions and arrows between circles and rectangles represent arcs whose direction is indicated by the tip. The current state $M$ of the Petri net is displayed by placing $M(p)$ dots in the circle representing the place $p$. Figure~\ref{net:sample} gives an example for the graphical representation of a Petri net.

\begin{figure}
	\centering
	\begin{tikzpicture}
	\node[place,tokens=2] (0) {};
	\node[transition] (1) [right of=0] {};
	\node[place]      (2) [right of=1] {};
	\path
		(0) edge [post] (1)
		(1) edge [post] (2);
	\end{tikzpicture}
	\caption{Graphical representation of a Petri net with two places and one transition. The current state is a marking that sends the left place to $2$ and the second place to $0$.}
	\label{net:sample}
\end{figure}

A transition $t$ is said to be \emph{active}, iff
$$\forall p\in P\colon (p,t)\in F\Rightarrow M(p) > 0$$
holds. In this case \emph{firing} $t$ means to transform the current state $M$ into $M'$ by the update rule
$$M'(p) \defeq \begin{cases}
M(p) - 1 & (p,t)\in F,\ (t,p)\notin F \\
M(p) + 1 & (p,t)\notin F,\ (t,p)\in F \\
M(p) & \text{else}
\end{cases}$$
for all $p\in P$. We interpret this as follows: For every incoming arc $(p,t)$ from $p$ a token in $p$ is consumed and for every outgoing arc $(t, p')$ to $p'$ a new token is placed into $p'$. Note that the notion of tokens ``moving through transitions'' instead of consumption and creation of tokens is erroneous as we will see later on when attaching data to them.

\begin{figure}
	\centering
	\begin{tikzpicture}
	\node[place,tokens=1] (0) {};
	\node[transition] (1) [right of=0] {};
	\node[place]      (2) [right of=1] {};
	\path
		(0) edge [post] (1)
		(1) edge [post] (2);
		
	\node[place,tokens=1] (3) [below of =0] {};
	\node[transition] (4) [right of=3] {};
	\node[place]      (5) [right of=4] {};
	\path
	(3) edge [post] (4)
	(4) edge [post] (5);
	\end{tikzpicture}
	\caption{Modelling concurrent processes in a Petri net.}
	\label{net:sample_concurrency}
\end{figure}

An important property of Petri nets is given by the fact that in many cases the same result is obtained if the order in which two transitions $t_1$ and $t_2$ fire is reversed. This allows the modelling of concurrent behaviour. When executing the Petri net model on a machine, the necessary steps for firing $t_1$ and $t_2$ respectively may be performed in parallel. Figure~\ref{net:sample_concurrency} depicts such a situation. Note that figure~\ref{net:sample} also shows an example for concurrent behaviour in which the transition may fire twice. However, the order in which the tokens in the left place are consumed is of no relevance. This example seems trivial as tokens located at the same place are indistinguishable. However, this is not the case anymore when attaching data to tokens. In fact, figure~\ref{net:sample} depicts the common scenario of concurrent behaviour in our algorithm, as parallel execution is mostly achieved by invoking the same routine for varying, independent input data.

\subsection*{Coloured Petri nets and guards}

Workflows describe a (possible concurrent) progression of processes, hence it is evident to model a process by a transition. The input and output data of a process is taken into account by attaching data values of a predefined type to each token. Then every incoming arc provides input data for the process corresponding to the target transition, whereas every outgoing arc describes output data of the source transition that should be put into the target place. A place may only hold tokens of a single type, introducing strong typing to Petri nets. This kind of Petri net is called \emph{coloured Petri net}, developed by Kurt Jensen in 1980. The term originates from the analogy of understanding each distinct type as a unique color. A formal treatment of this concept may be found in \cite{cp_nets}.

In order to control the data flow, each transition may be tagged with a predicate over the types of places with incoming arcs, which is called \emph{condition} or \emph{guard} \cite{cp_nets}. The activation rule for a transition is extended such that not only a token has to be present for every incoming arc, but also the data values of the consumed tokens have to satisfy the condition. \gpispace{} provides its own expression language in order to specify conditions. Figure~\ref{net:sample_guard} shows a transition that consumes tokens with positive data values only.

\begin{figure}
	\centering
	\begin{tikzpicture}
	\node[place,tokens=2] (0) {};
	\node[transition] (1) [right of=0] {};
	\path
		(0) edge [post] (1);
	\node[anchor=west] at ($(0) + (-1.8,0)$) {$\begin{aligned}x &= 1\\[-3\jot] x&= -1\end{aligned}$};
	\node at ($(1) + (0,0.7)$) {$\gpiresolve{x} \gpigt 0$};
	\end{tikzpicture}
	\caption{Guards in a Petri net. Only the token with $x=1$ activates the transition.}
	\label{net:sample_guard}
\end{figure}

\subsection*{Expressions}

\gpispace{} supports two kinds of transitions. The first kind triggers a module call whenever firing, executing arbitrary \cplusplus code where the input and output tokens are mapped to references. The module call is packaged as a job and scheduled by the \ac{DRTS}. A lightweight approach that requires no scheduling is given by the second kind of transition. The expression language provided by \gpispace{} allows to specify the output data values of a transition by means of simple operations on the input data values. This extension is intended to be used for minor computations and saves the overhead of generating and scheduling a job by executing the expression directly within the \ac{WE}. For instance, counting consumed tokens may be realised that way, see figure~\ref{net:sample_expression}. If the counting would be realised with \cplusplus, each increment will cause a disproportionate communication and scheduling overhead.

\begin{figure}[t]
	\centering
	\begin{tikzpicture}
	\node[place,tokens=3] (0) {};
	\node[transition] (1) [right of=0] {};
	\node[place,tokens=1] (2) [right of=1] {};
	\path
		(0) edge [post] (1)
		(2) edge [pre, bend right] (1)
		(2) edge [post, bend left] (1);
	\node[draw] at ($(1) + (0,-0.9)$) {$\gpiresolve{i} \defeq \gpiresolve{i} + 1$};
	\node at ($(2) + (1,0)$) {$i = 0$};
	\end{tikzpicture}
	\caption{Expressions in a Petri net. After firing the transition three times, the left place contains no tokens anymore and the right place contains one token with $i=3$.}
	\label{net:sample_expression}
\end{figure}

\subsection*{Executing Petri nets}

The \ac{WE} is responsible for executing a given Petri net. It manages its current state and identifies active transitions. Whenever such a transition is detected, its input tokens are consumed, bundling the attached data values into a job that describes the execution of the transition. Then, this job is submitted to the \ac{DRTS} where it awaits its execution. The set of active transitions not being executed already is called front of activities and generally contains numerous elements at once due to the locality of Petri nets. Thus, the \ac{DRTS} is able to utilise the available capacities by scheduling multiple jobs in parallel.

When a job finishes, the results are returned to the \ac{WE} which extracts the output tokens and updates the state of the Petri net accordingly. Note that during the whole process described above, the input tokens for an active transition are consumed whereas the output tokens are not available yet. In order to fully describe the state of an executable Petri net, one has to incorporate the knowledge about all active transitions and its cached tokens into the state description. Hence, the \ac{WE} has to manage \emph{timed}, coloured Petri nets. Figure~\ref{net:sample_execution} illustrates the process of executing two active transitions in parallel. The module called by the transition sleeps for $x$ seconds, where $x$ is the input data value, and then returns the increment $x + 1$.

\begin{figure}[t]
	\centering
	\begin{tabular}{cc}
		\adjustbox{valign=t}{\begin{tikzpicture}[framed]
		\node[place,tokens=2] (0) {};
		\node[transition] (1) [right of=0] {};
		\node[place]      (2) [right of=1] {};
		\path
		(0) edge [post] (1)
		(1) edge [post] (2);
		\node[anchor=west] at ($(0) + (-1.6,0)$) {$\begin{aligned}x &= 1\\[-3\jot] x&= 5\end{aligned}$};
		\node[anchor=west] at ($(2) + (0.4,0)$) {\phantom{$x = 1$}};
		
		\node[anchor=west,inner sep=0] at ($(0.west) + (0,-0.8)$) {active transitions:};
		\end{tikzpicture}}
		&
		\adjustbox{valign=t}{\begin{tikzpicture}[framed]
		\node[place,tokens=1] (0) {};
		\node[transition] (1) [right of=0] {};
		\node[place]      (2) [right of=1] {};
		\path
		(0) edge [post] (1)
		(1) edge [post] (2);
		\node[anchor=west] at ($(0) + (-1.6,0)$) {$x = 1$};
		\node[anchor=west] at ($(2) + (0.4,0)$) {\phantom{$x = 1$}};
		
		\node[anchor=west,inner sep=0] at ($(0.west) + (0,-0.8)$) {active transitions:};
		\node[place,tokens=1] (0') at ($(0) + (0,-1.6)$) {};
		\node[transition] (1') [right of=0'] {};
		\node[place]      (2') [right of=1'] {};
		\path
		(0') edge [post] (1')
		(1') edge [post] (2');
		\node[anchor=west] at ($(0') + (-1.6,0)$) {$x = 5$};
		\end{tikzpicture}}
		\\[2.5cm]
		\adjustbox{valign=t}{\begin{tikzpicture}[framed]
		\node[place] (0) {};
		\node[transition] (1) [right of=0] {};
		\node[place]      (2) [right of=1] {};
		\path
		(0) edge [post] (1)
		(1) edge [post] (2);
		\node[anchor=west] at ($(0) + (-1.6,0)$) {\phantom{$x = 1$}};
		\node[anchor=west] at ($(2) + (0.4,0)$) {\phantom{$x = 1$}};
		
		\node[anchor=west,inner sep=0] at ($(0.west) + (0,-0.8)$) {active transitions:};
		\node[place,tokens=1] (0') at ($(0) + (0,-1.6)$) {};
		\node[transition] (1') [right of=0'] {};
		\node[place]      (2') [right of=1'] {};
		\path
		(0') edge [post] (1')
		(1') edge [post] (2');
		\node[anchor=west] at ($(0') + (-1.6,0)$) {$x = 5$};
		\node[place,tokens=1] (0'') [below of=0'] {};
		\node[transition] (1'') [right of=0''] {};
		\node[place]      (2'') [right of=1''] {};
		\path
		(0'') edge [post] (1'')
		(1'') edge [post] (2'');
		\node[anchor=west] at ($(0'') + (-1.6,0)$) {$x = 1$};
		\end{tikzpicture}}
		&
		\adjustbox{valign=t}{\begin{tikzpicture}[framed]
		\node[place] (0) {};
		\node[transition] (1) [right of=0] {};
		\node[place,tokens=1] (2) [right of=1] {};
		\path
		(0) edge [post] (1)
		(1) edge [post] (2);
		\node[anchor=west] at ($(0) + (-1.6,0)$) {\phantom{$x = 1$}};
		\node[anchor=west] at ($(2) + (0.4,0)$) {$x = 2$};
			
		\node[anchor=west,inner sep=0] at ($(0.west) + (0,-0.8)$) {active transitions:};
		\node[place,tokens=1] (0') at ($(0) + (0,-1.6)$) {};
		\node[transition] (1') [right of=0'] {};
		\node[place]      (2') [right of=1'] {};
		\path
		(0') edge [post] (1')
		(1') edge [post] (2');
		\node[anchor=west] at ($(0') + (-1.6,0)$) {$x = 5$};
		\end{tikzpicture}}
		\\[3.8cm]
		\adjustbox{valign=t}{\begin{tikzpicture}[framed]
		\node[place] (0) {};
		\node[transition] (1) [right of=0] {};
		\node[place,tokens=2] (2) [right of=1] {};
		\path
		(0) edge [post] (1)
		(1) edge [post] (2);
		\node[anchor=west] at ($(0) + (-1.6,0)$) {\phantom{$x = 1$}};
		\node[anchor=west] at ($(2) + (0.4,0)$) {$\begin{aligned}x &= 2\\[-3\jot] x&= 6\end{aligned}$};
		
		\node[anchor=west,inner sep=0] at ($(0.west) + (0,-0.8)$) {active transitions:};
		\end{tikzpicture}}
		&
		\\
	\end{tabular}
	\caption{Possible execution of a Petri net. The module called by the transition sleeps for $x$ seconds and then returns the increment $x + 1$.}
	\label{net:sample_execution}
\end{figure}

The \ac{WE} itself is executed on a single node of the underlying system. Hence, special care must be taken when designing Petri nets. Instead of firing numerous transitions with execution times of milliseconds, one is advised to aggregate these transitions into a single one with an execution time of several seconds. Otherwise, the \ac{WE} quickly becomes a bottleneck since the amount of scheduled transitions per second does not scale with the number of available computing nodes. Furthermore, the marking of the Petri net is stored in the RAM of a single node. If large quantities of data have to be managed, the tokens should only contain a reasonable amount of metadata pointing to the real data which -- for instance -- is stored in a shared directory.

\gpispace{} assumes that module calls have no side effects. Thus, when implementing the execution code for the transitions, possible racing conditions and concurrency issues have to be considered properly. Furthermore, each worker managed by the \ac{DRTS} is executed in its own process, but beyond that, jobs assigned to the same worker run in the same environment. For this reason jobs are influenced by changes of global state by prior executions. This behaviour causes several issues when integrating \singular{}, which is implemented as a large, global state machine.

\section{Integration of Singular}
\singular{} is a computer algebra system for polynomial computations, with special emphasis on commutative and non\=/commutative, algebraic geometry and singularity theory \cite{singular}. Typically, algorithms utilising \singular{} are written in its own \c\=/like programming language, which is interpreted at runtime. Besides providing a console frontend for code execution, \singular{} also comes with the \c{} library \libsingular{} that implements its full backend. In particular, \c{} methods for interpreting and executing \singular{} code are available. \libsingular{} also exposes the internal data structures used by \singular{}, so that one can write conversion methods between \singular{} types and user defined types, i.e. \gpispace{} types.

Since module calls in \gpispace{} permit the execution of arbitrary \cplusplus{} code, every legacy application with a \c{} interface such as \singular{} can easily be integrated and executed. In order to execute \singular{} code in a module call, one has to perform the following steps:
\begin{enumerate}
	\item Check if a \singular{} instance is already running on the current worker. If not, initialise \singular{}.
	\item Convert \gpispace{} types to \singular{} types.
	\item Map the converted data to identifiers in order to address it by \singular{} code.
	\item Formulate a \c{} string containing the \singular{} code to be executed and pass it to the \singular{} interpreter via \libsingular{}. Note that instead of hard coding large blocks of code, it is reasonable to outsource them into a \singular{} library. Then, this step reduces to loading the library and invoking a single method.
	\item Extract the data mapped to all identifiers that contain results of the computation.
	\item Convert the extracted \singular{} types to \gpispace{} types that can be attached to tokens.
\end{enumerate}

\subsection*{Drawback: Global state}
Although integrating \singular{} code is sufficiently easy, the approach described above suffers from a serious issue that originates from \singular{} heavily relying on global state. For instance, loaded libraries, declared variables and the current basering are managed in global data structures. If module calls are supposed to have no side effects, all changes concerning global state have to be reverted before returning from them. Unfortunately, \libsingular{} lacks the feature of creating and restoring images of its global state. In fact, there seems to be no way of resetting \singular{} to its initial state short of killing the current process, which -- obviously -- is not desirable.

Not being able to reset the \singular{} instance also makes writing decoupled test cases more problematic. In order to avoid tests being influenced by one another due to global \singular{} state, each test has to be executed in a separate process. Unfortunately, this prevents the utilisation of many \gtest{} features such as bundling test cases, reusing data configurations and parametrising tests.

\subsection*{Drawback: No in memory serialization of \singular{} types}
When \gpispace{} has to share \singular{} data between module calls without knowing the internal structure of the data, this can be achieved by serialising the data into \acp{BLOB} which then are managed by the \ac{WE}. \singular{} supports the serialization of its common data types by using \emph{ssi file links}. As the name suggests, the serialization string always is written to a file. It would be desirable to also obtain a serialization string in memory without the overhead of using virtual files or reading the created ssi files. Currently, we pass metadata (that is file paths) pointing to ssi files in a shared directory instead of acquiring and passing \acp{BLOB}.

\subsection*{Drawback: \singular{} programming language lacking data structure implementations}
Since the purpose of developing a parallel implementation of algorithm~\ref{algorithm:main} is the computation of large examples within a reasonable timeframe, it is crucial to choose optimal data structures for large sets of data. Unfortunately, the \singular{} programming language lacks tree\=/like data structures with $\mathcal{O}(n)$ characteristics and, more importantly, hash tables with average time complexity of $\mathcal{O}(1)$ for insertion, searching and deletion. For this reason, all tasks dealing with the detection of duplicates, were moved from \singular{} into the \cplusplus{} environment, where the \ac{STL} provides all necessary data structure implementations. Hence, the set of found GIT cones is managed in \cplusplus{}. The elimination of duplicates in orbit computations as well as the computation of the symmetry group action on GIT cones have been reworked in \cplusplus{}, resulting in a substantial performance gain compared to the \gitfanlib{} implementation.


\section{Encoding GIT cone orbits}
\label{sec:git_cone_management}
In chapter~\ref{chap:algorithm}, we suggested to traverse a graph that reflects the structure of the maximal cones located in the GIT fan. Since nodes of the graph may be reachable by multiple paths, the traversal algorithm has to manage a list of found nodes such that no node is expanded twice. For this reason it is crucial to develop an encoding of nodes, i.e. GIT cone orbits, that 
\begin{enumerate}[label={\upshape(\roman*)}]
	\item conserves memory, permitting fast bit\=/per\=/bit comparisons,
		\label{enum_item:git_cone_orbit_property_memory}
	\item allows to recover the encoded object from the code word,
		\label{enum_item:git_cone_orbit_property_invertible}
	\item does not require additional effort when computing code words and
		 \label{enum_item:git_cone_orbit_property_computation_effort}
	\item is easily accessible for operations that are frequently performed on the nodes.
		\label{enum_item:git_cone_orbit_property_operations}
\end{enumerate}

\subsection*{Encoding GIT cones}
In order to encode GIT cone orbits, we require an appropriate encoding for GIT cones that allows us to apply the group action of $\mathcal{S}$ with minimal costs. The encoding described here originates from \cite[Construction 4.3]{gitfan_symmetry} and has been implemented in \gitfanlib{}. It utilises the result of \todo{orbits mit p enthalten git cone}, stating that every full dimensional GIT cone is the finite intersection of all full dimensional orbit cones containing the GIT cone. Thus, every GIT cone $\lambda$ can be uniquely identified by bits indexed over $\Omega_\mathfrak{a}^{(k)}$ such that the bit indexed by $\vartheta\in\Omega_\mathfrak{a}^{(k)}$ is set to $1$ iff $\lambda \subseteq \vartheta$. Formally:
$$\code\colon \Lambda(\mathfrak{a},Q)(k) \rightarrow \{0,1\}^{\Omega_\mathfrak{a}^{(k)}},\quad
\lambda \mapsto (b_\vartheta)_{\vartheta\in\Omega_\mathfrak{a}^{(k)}}\quad\text{with}\quad
b_\vartheta = 
\begin{cases}
1, & \vartheta\subseteq\lambda \\
0, & \vartheta\not\subseteq\lambda
\end{cases}\ .
$$
Obviously, this mapping is injective and yields an encoding with code words of length $|\Omega_\mathfrak{a}^{(k)}|$, guaranteeing property \ref{enum_item:git_cone_orbit_property_memory}. The GIT cone is recovered by intersecting all orbit cones indexing a $1$ in the code word. Hence, \ref{enum_item:git_cone_orbit_property_invertible} holds.

Next, we verify property \ref{enum_item:git_cone_orbit_property_computation_effort}. Since every GIT cone $\lambda(p)$ occurring in the algorithm is constructed by taking the relative inner point $p$ and intersecting all orbit cones containing $p$, the code word can be computed alongside the construction of $\lambda(p)$ due to \todo{orbits mit p enthalten git cone}.

Finally, verifying \ref{enum_item:git_cone_orbit_property_operations},  we have to check that the group action of $\mathcal{S}$ on the code words defined by
$$\sigma\cdot\code(\lambda) \defeq \code(\sigma\cdot\lambda) = \code(A_\sigma\cdot\lambda)$$
is realised by a low\=/cost operation on bit strings. Since we have
$$\lambda\subseteq\vartheta\ \Leftrightarrow\ A_\sigma\cdot\lambda\subseteq A_\sigma\cdot\vartheta,$$ the group action simply permutes the characters of code words in the same way as $\mathcal{S}$ acts on $\Omega_\mathfrak{a}^{(k)}$, see \todo{Referenz: S wirkt auf OrbitCones}. We are going to compute the group action $\mathcal{S}\acts\Omega_\mathfrak{a}^{(k)}$ in the next section, so that it is available during the traversal process. Hence, applying a group element to a code word is implemented as the application of a permutation which is stored in a lookup table.

\subsection*{Extending the encoding to orbits}
After encoding GIT cones, we are able to encode GIT cone orbits by mapping each orbit to the code word of an uniquely identified representative. We choose the representative with the minimal code word regarding the lexicographical order over the alphabet $\{0,1\}$. Hence, the encoding is given by
$$\code\colon \faktor{\Lambda(\mathfrak{a},Q)(k)}{\S} \rightarrow \{0,1\}^{\Omega_\mathfrak{a}^{(k)}},\quad
\O \mapsto \min\limits_{\leq_\text{lex}}\ \{\code(\lambda) \mid \lambda\in\O\}$$
and clearly is injective. Property \ref{enum_item:git_cone_orbit_property_memory} is inherited from the GIT cone encoding. The same applies to property \ref{enum_item:git_cone_orbit_property_invertible}, since we already presented an approach for applying group elements to code words, so that we are able to recover the orbit from a single representative. Property \ref{enum_item:git_cone_orbit_property_computation_effort} holds since orbits typically are constructed from a representative and finding the minimal representative is cheap due to property \ref{enum_item:git_cone_orbit_property_operations} of the GIT cone encoding. Finally, since operations on orbits usually are given by well defined operations on their representatives, \ref{enum_item:git_cone_orbit_property_operations} also is inherited from the GIT cone encoding. One simply applies the operation on the GIT cone that is represented by the code word of the orbit.

\section{Application flow \& Concurrency}

We differentiate between three stages when implementing the algorithm described in chapter~\ref{chap:algorithm}. During the first stage, preprocessing of the input data and the identification of all orbits of \afaces{} with full dimensional image takes place. The second stage covers the computation of the orbit cones and the symmetry group action on GIT cone\=/hashes. This data is mandatory when determining the GIT fan in the third stage by traversing all maximal cones of the fan. Figure~\ref{net:algorithm_overview} depicts the simplified data flow of the algorithm in which each stage is collapsed into a single transition. Dashed arcs indicate read-only connections, that is data values of tokens in the source place are passed to the target transition without consuming them when it fires. Hence, \emph{read-only} connections give access to static data computed once (such as processed input data), without discarding it due to consumption.

\begin{figure}[t]
	\centering
	\begin{tikzpicture}
	\node[place] (raw) {raw input};
	\node[transitiontext] (p1) at ($(raw) + (0,-1.5)$) {Stage 1};
	\node[place,anchor=east,minimum width=4cm] (data) at ($(p1) + (-0.5,-1.5)$) {problem data};
	\node[place,anchor=west,minimum width=4cm] (afaces) at ($(p1) + (0.5,-1.5)$) {\afaces};
	\node[transitiontext] (p2) at ($(p1) + (0,-3)$) {Stage 2};
	\node[place,anchor=east,minimum width=4cm] (action) at ($(p2) + (-0.5,-1.5)$) {group action};
	\node[place,anchor=west,minimum width=4cm] (oc) at ($(p2) + (0.5,-1.5)$) {orbit cones};
	\node[transitiontext] (p3) at ($(p2) + (0,-3)$) {Stage 3};
	\node[place] (gc) at ($(p3) + (0,-1.5)$) {GIT cones};
	
	\node[transitiontext,anchor=west] (oc_output) at ($(oc.east) + (1,0)$) {output};
	\node[transitiontext,anchor=west] (gc_output) at ($(oc_output.west) + (0,-3)$) {output};
	
	\path
	(raw) edge [post] (p1)
	(p1) edge [post] (data)
	(p1) edge [post] (afaces)
	(data) edge [dashed,post] (p2)
	(afaces) edge [post] (p2)
	(p2) edge [post] (action)
	(p2) edge [post] (oc)
	(action) edge [post] (p3)
	(oc) edge [dashed,post] (p3)
	(data) edge [out=-135,in=90,dashed] ($(action.west)+(-0.5,0)$)
	($(action.west)+(-0.5,0)$) edge [out=-90,in=180,dashed,post] (p3)
	(p3) edge [post] (gc)
	(oc) edge [dashed,post] (oc_output)
	(gc) edge [dashed,post] (gc_output);
	\end{tikzpicture}
	\caption{The data flow of the overall algorithm modelled as a Petri net with collapsed stages.}
	\label{net:algorithm_overview}
\end{figure}

We reuse the notation from chapter~\ref{chap:algorithm}. Recall that the matrix $Q\in\field^{k\times r}$ of rank $r$ with columns $q_1,\dots,q_r$ encodes a torus action on the affine variety $X = V(\ideal)\subseteq \field^r$ where $\ideal\subseteq \field[x_1,\dots,x_r]$ is a homogeneous ideal with respect to the grading $\deg(x_i) = q_i$, $1\leq i\leq r$. The positive orthant $\mathbb{Q}_{\geq 0}^r$ is denoted by $\gamma$. Furthermore, $\mathcal{S}$ (possibly trivial) denotes a symmetry group of the action of $Q$ on $X$. \todo{Notation mit Kap. 3 prüfen}

\subsection*{Stage 1: Preprocessing, \afaces}
Stage 1 starts with preprocessing the raw input supplied by the user. During the process, we compute the following objects sequentially:
\begin{itemize}
	\item The amount $r$ of variables occurring in the basering and the dimension $k$ of the ambient space in which the GIT fan lives. Both values are extracted from the matrix describing the torus action.
	\item the moving cone (optionally, see section~\ref{sec:additional_features})
	\item orbit representatives for the faces of the simplex $\gamma$ under the symmetry group $\mathcal{S}$ in case it is nontrivial and no set of precomputed orbit representatives is provided (see section~\ref{sec:additional_features}).
\end{itemize}

In a next step, the set of orbit representatives for the faces of the simplex $\gamma$ is partitioned into parts of equal size. A token is generated for every part, which then is processed in parallel, selecting only the \afaces{} with full dimensional image under $Q$. Finally, the reduced parts are merged into a single token that then is passed to the next stage. The Petri net modelling this process is depicted in figure~\ref{net:algorithm_stage_1}.

Since the [partition]\=/transition has to modify a state token which describes the faces assigned to a part already, it may execute sequentially only. This behaviour is enforced by adding the [partitioner\=/state]\=/place with a token describing the initial state of the partitioning process.
The [\afaces]\=/place is initialised with a token containing an empty list. Since the [merge]\=/transition is synchronized via this token, it also may execute sequentially only. However, the computationally heavy task of identifying \afaces{} is represented by the [select \afaces{}]\=/transition, which is not synchronized. Therefore, any amount of faces may be processed simultaneously.

\begin{figure}[t]
	\centering
	\begin{tikzpicture}
	\node[placehighlight] (raw) {raw input};
	\node[transitiontext,anchor=west] (preprocess) at ($(raw.east) + (2,0)$) {preprocess};
	\node[placehighlight] (data) at ($(preprocess) + (0,-1.5)$) {problem data};
	\node[transitiontext] (partition) at ($(data) + (0,-1.5)$) {partition};
	\node[place,anchor=east] (partitioner-state) at ($(partition.west) + (-1,0)$) {partitioner-state (ps)};
	\node[place] (faces) at ($(partition) + (0,-1.5)$) {faces-subset};
	\node[transitiontext] (select) at ($(faces) + (0,-1.5)$) {select \afaces};
	\node[place] (afaces)  at ($(select) + (0,-1.5)$) {\afaces-subset};
	\node[transitiontext] (merge)  at ($(afaces) + (0,-1.5)$) {merge};
	\node[placehighlight,anchor=west] (all-afaces)  at ($(merge.east) + (2,0)$) {\afaces};
	
	\node[anchor=west] at ($(partition.east) + (0.1,0)$) {$\gpiresolve{\text{ps.finished}}$};
	\node[anchor=north,draw] at ($(merge.south) + (0,-0.25)$) {$\gpiresolve{\text{\afaces}} \defeq \texttt{stack\_join}(\gpiresolve{\text{\afaces}}, \gpiresolve{\text{\afaces-subset}})$};
	
	\path
	(raw) edge [post] (preprocess)
	(preprocess) edge [post] (data)
	(data) edge [dashed,post] (partition)
	(data) edge [out=0,in=0,looseness=2,dashed,post] (select)
	(partition) edge [bend right,post] (partitioner-state)
	(partitioner-state) edge [bend right,post] (partition)
	(partition) edge [post] (faces)
	(faces) edge [post] (select)
	(select) edge [post] (afaces)
	(afaces) edge [post] (merge)
	(merge) edge [bend left,post] (all-afaces)
	(all-afaces) edge [bend left,post] (merge);
	\end{tikzpicture}
	\caption{Stage 1 of the algorithm. After preprocessing the input and partitioning all faces, each part is reduced and merged into a single token. Highlighted places correspond to places in figure~\ref{net:algorithm_overview}.}
	\label{net:algorithm_stage_1}
\end{figure}

\subsection*{Stage 2: Orbit cones, group action on GIT cone\=/hashes}
After identifying a representative system for the \afaces{} with full dimensional image, we are able to determine all full dimensional orbit cones by taking the image under $Q$ and computing orbits under the symmetry group $\mathcal{S}$ acting on $Q(\gamma)$. The following steps are performed in successive order:

\begin{enumerate}
	\item Compute the cones $Q(\gamma_0)$ for all \afaces{} $\gamma_0$ returned by stage 1. These images contain a representative system for the set of full dimensional orbit cones $\Omega_\mathfrak{a}^{(k)}$. \todo{Notation bzgl. Kapitel 3 anpassen + Verweis auf Beweis der Aussage}
	\item Optionally (see section~\ref{sec:additional_features}): Intersect all obtained cones with the moving cone. Since the moving cone is invariant under the symmetry group $\mathcal{S}$ \todo{Referenz}, the resulting cones contain a representative system for $\Omega_\mathfrak{a}^{(k)}$ intersected with the moving cone.
	\item For each obtained cone $\sigma$: Compute the finite orbit $\mathcal{S}\sigma$ and the group action of $\mathcal{S}$ on $\mathcal{S}\sigma$.
	\item Eliminate duplicate orbits so that the remaining orbits form a partition of $\Omega_\mathfrak{a}^{(k)}$. \label{enum_item:eliminate_duplicate_orbits}
	\item Aggregate all orbits into a single list containing $\Omega_\mathfrak{a}^{(k)}$. Utilise the previously determined group actions on the orbits in order to construct the group action of $\mathcal{S}$ on $\Omega_\mathfrak{a}^{(k)}$. \label{enum_item:aggregate_orbits}
\end{enumerate}

With the exception of step~\ref{enum_item:aggregate_orbits}, in all steps a prescribed operation has to be executed on each element of a list of either cones or orbits. Hence, we are able to parallelise the computations in each step by partitioning the list into equally sized parts as described for stage 1. Note that the elimination of duplicates in step~\ref{enum_item:eliminate_duplicate_orbits} translates to selecting all list entries such that no subsequent list entry equals the selected one.

The most computationally intensive operation is performed in step 3. Every group element is applied to the representative. Since the stabiliser subgroup may be nontrivial, we have to take care of possible duplicates. In order to do so, whenever a new cone is determined, the \gitfanlib{} implementation runs expensive comparisons with all previously found orbit elements and discards the cone if any matches are detected. We reworked the approach by
exchanging the expensive comparison of cones with a string comparison of the cone's serialization, which is one to one for cones returned from \singular's \texttt{canonicalizeCone} routine. Furthermore, we manage the found cones in a hash table \todo{Referenz zu einer Hashtable-Einführung?}, using a hash function on the serialization string. Since the maximal orbit size is known to be the cardinality of $\mathcal{S}$, we can adjust the size of the hash table such that hash collisions are unlikely. Hence, the duplicate elimination comes for free by performing only one string comparison on average.

The \gitfanlib{} implementation also compares cones in order to determine the action of $\mathcal{S}$ on the git cone orbits. However, the action may be constructed from the cayley table of $\mathcal{S}$ as long it is known which group elements map to the same orbit element. For this reason we drop the cone comparisons and compute the action by composing permutations in $\mathcal{S}$, which is a very cheap operation scaling with the amount $r$ of variables occurring in the basering..

\subsection*{Stage 3: GIT fan traversal}

\todo{Formalen Teil zum Algorithmus nach Kapitel 3 auslagern}

Now we are ready to determine the GIT fan. As described in chapter~\ref{chap:algorithm}, this can be achieved by successively computing neighbours of full dimensional GIT cones. Note that the support of the GIT fan is convex, thus connected. From an algorithmic point of view, the GIT fan traversal translates to a graph traversal with the GIT cones being the nodes and the property of two GIT cones sharing a common facet being the edges. Formally, we consider the undirected, connected Graph $G_{\langle e \rangle}=(V_{\langle e \rangle},E_{\langle e \rangle})$ with
\begin{align*}
V_{\langle e \rangle} &\defeq \Lambda(\mathfrak{a},Q)(k),\\
E_{\langle e \rangle} &\defeq \left\{\{\lambda_1, \lambda_2\} \mid \lambda_1, \lambda_2 \in \Lambda(\mathfrak{a},Q)(k) \text{ and }\lambda_1 \cap \lambda_2 \text{ is a facet of both }\lambda_1\text{ and }\lambda_2\right\}.
\end{align*}

Taking the symmetry group $\mathcal{S}$ into account, it suffices to traverse the graph $G_\mathcal{S} = (V_\mathcal{S}, E_\mathcal{S})$ with
\begin{align*}
V_\mathcal{S} &\defeq \faktor{\Lambda(\mathfrak{a},Q)(k)}{\mathcal{S}},\\
E_\mathcal{S} &\defeq \left\{\{\mathcal{S}\lambda_1, \mathcal{S}\lambda_2\} \mid \{\lambda_1, \lambda_2\} \in  E_{\langle e \rangle}\right\}.
\end{align*}

Since $G_{\langle e \rangle}$ is connected, it immediately follows that $G_\mathcal{S}$ is connected as well.

\todo{Ende formaler Teil}

In order to traverse any undirected, connected graph $G=(V,E)$, the following has to be provided:

\begin{itemize}
	\item A computable encoding of $V$, that is an injective function mapping $V$ to binary words.
	\item A routine \texttt{START} that computes an initial node $v_0\in V$.
	\item A routine \texttt{NEIGHBOURS} computing the function $\mathcal{N}\colon V \rightarrow \mathcal{P}(V)$ which lists all neighbours of a node $v$, that is $f(v) = \{w \mid \{v,w\}\in E\}$.
\end{itemize}

In our case, we use the GIT cone orbit encoding described in section~\ref{sec:git_cone_management}. \texttt{START} is implemented as in \gitfanlib{}, that is selecting random points $p\in Q(\gamma)$ until the corresponding GIT cone $\lambda(p)$ \todo{Notation konsistent?} has full dimension. Then, \texttt{START} returns the GIT cone orbit\=/code of $\mathcal{S}\lambda(p)$. The routine \texttt{NEIGHBOURS} is given by algorithm~\ref{algo:git_cone_orbitneighbours}.

It remains to model a Petri net describing a parallel graph traversal. Note that this model is applicable for any kind of fan traversals, e.g. in the context of tropical varieties \todo{Referenz von Janko}, as long as the above mentioned demands are satisfied. 







\section{Input \& Output structure}

\section{Additional features}
\label{sec:additional_features}
an Kommandozeilenparameter entlanghangeln

\section{Software testing}