\chapter{Useful lemmata}

\begin{lemmaApp}
	\label{lemma:convex_fan_maximal_cones}
	Let $\Sigma\subseteq\mathbb{R}^k$ be a $k$-dimensional fan with convex support. Then every cone in $\Sigma$ is a face of a cone in $\Sigma^{(k)}$. In particular, $\Sigma$ is uniquely determined by $\Sigma^{(k)}$.
\end{lemmaApp}
\begin{proof}
	Let $\tau\in\Sigma$ and choose a point $p\in\relint(\tau)$. We select hyperplanes $H_\theta$ for all $\theta\in\Sigma\setminus\Sigma^{(k)}$ such that $\theta\subseteq H_\theta$. Then we have
	$$S \defeq |\Sigma| \setminus \left(\bigcup_{\theta\in\Sigma\setminus\Sigma^{(k)}} H_\theta\right) \subseteq \bigcup_{\sigma\in\Sigma^{(k)}} \sigma \eqdef |\Sigma^{(k)}|.$$
	Since $|\Sigma|$ has dimension $k$ and $S$ arises from $|\Sigma|$ by removing finitely many hyperplanes, $S$ has dimension $k$. For this reason we find $q\in S \setminus\{p\}$. By Bézout's theorem, every hyperplane $H_\theta$ intersects the line through $p$ and $q$ in at most one point. As $|\Sigma|$ is convex, the connecting line $\overline{pq}$ is contained in $|\Sigma|$ and $\overline{pq} \cap S$ emerges from $\overline{pq}$ by removing a finite amount of points. Hence, we find a sequence $(p_n)_{n\in\natural}\subseteq \overline{pq} \cap S$ such that $\lim\limits_{n\rightarrow\infty} p_n = p$. It follows that $p\in\overline{S}$. Since $|\Sigma^{(k)}|$ is closed as a finite union of closed sets, $p\in|\Sigma^{(k)}|$ holds. Let $\sigma\in\Sigma^{(k)}$ such that $p\in\sigma$. Then the common face of $\sigma$ and $\tau$ contains $p$ and thus has to be $\tau$. We conclude that $\tau\preceq\sigma$.
	
	Now, $\Sigma$ is uniquely determined by $\Sigma^{(k)}$ since fans are closed under taking faces.
\end{proof}

\chapter{Ideal quotients}
\label{appendix:ideal_quotients}

Theory about saturated ideals allows us to implement an efficient monomial containment test, see \cite{gitfan_symmetry}. Here, we establish some elementary properties that are used in this thesis.

\begin{defiApp}
	Let $I,J$ be ideals over a commutative ring $R$. Then we define the \emph{ideal quotient} $(I:J)$ by
	$$(I:J) = \{r\in R\ |\ rJ\subseteq I\}.$$
	The \emph{saturation} $(I:J^\infty)$ is given by
	$$(I:J^\infty) = \bigcup_{n\in \natural}(I:J^n).$$
\end{defiApp}

\begin{propApp}
	Let $I$, $J$, $K$ be ideals over $R$ and let $f,g\in R$. Then we have:
	\begin{enumerate}[label={\upshape(\roman*)}]
		\item $I\subseteq J\ \Rightarrow\ (I:K^\infty)\subseteq(J:K^\infty)$,
		\label{enum_item:saturation_inclusion_property}
		\item $((I:(f)^\infty):(g)^\infty) = (I:(fg)^\infty)$.
		\label{enum_item:saturation_principal_property}
	\end{enumerate}
\end{propApp}
\begin{proof}
	\ref{enum_item:saturation_inclusion_property} immediately follows from the definition. For \ref{enum_item:saturation_principal_property}, note that
	$$(I:(f)) = \{r\in R\ |\ rf\in I\}\quad\text{and}\quad (I:(f)^\infty) = \{r\in R\ |\ \exists n\in\natural: rf^n\in I\}.$$
	
	Let $r\in (I:(fg)^\infty)$. We find $n\in \natural$ such that $(rg^n)f^n = r(fg)^n\in I$. Hence, $rg^n\in (I:(f)^\infty)$. It follows that $r\in((I:(f)^\infty):(g)^\infty)$.
	
	Conversely, let $r\in((I:(f)^\infty):(g)^\infty)$. Then we find $n\in \natural$ such that $rg^n\in (I:(f)^\infty)$. This implies that there exists an $m\in\natural$ such that $rg^nf^m\in I$. For this reason $r(fg)^{\max\{n,m\}}\in I$ and thus $r\in (I:(fg)^\infty)$ holds, proving \ref{enum_item:saturation_principal_property}.
\end{proof}